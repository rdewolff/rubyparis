%%%%%%%%%%%%%%%%%%%%%%%%%%%%%%%%%%%%%%%%%%%%%%%%%%%%%%
% Auteurs : 			Romain de Wolff & Bruno da Silva
% Date de création : 	2 janvier 2008
%%%%%%%%%%%%%%%%%%%%%%%%%%%%%%%%%%%%%%%%%%%%%%%%%%%%%%

\documentclass[10pt,a4paper,titlepage]{article}
\usepackage[utf8]{inputenc}     % encodage des characteres en utf8
\usepackage[francais]{babel} % pour la table des matières en français 

\usepackage{url} % pour les liens internet
\usepackage[colorlinks=true,linkcolor=black,bookmarks=true,bookmarksopen=true]{hyperref} % rendre les liens clickable
\usepackage{fancyhdr}	 

\usepackage[dvips]{graphicx}
\usepackage{epstopdf}


%%%%%%%%%%%%%%%%%%%%%%%%%%%%%%%%%%%%%%%%%%%%%%%%%%%%%%
% Pour l'utilisation de code
%%%%%%%%%%%%%%%%%%%%%%%%%%%%%%%%%%%%%%%%%%%%%%%%%%%%%%

\usepackage{listings} 
%\lstset{language=Java, breaklines, fontadjust, inputencoding=utf8, basicstyle=\small, numbers=left, numberstyle=\tiny, tabsize=2}

\usepackage{courier}
\usepackage{color}

% color definitions
\definecolor{dkgreen}{rgb}{0,0.6,0}
\definecolor{gray}{rgb}{0.5,0.5,0.5}
\definecolor{lightblue}{rgb}{0.92,0.92,1}

\lstset{language=Html,
  %keywords={break,case,catch,continue,else,elseif,end,for,function,
  %   global,if,otherwise,persistent,return,switch,try,while},
  keywords={script, document, function},
  basicstyle=\ttfamily\small,
  keywordstyle=\color{blue},
  commentstyle=\color{dkgreen},
  stringstyle=\color{red},
  numbers=none,
  numberstyle=\tiny\color{gray},
  stepnumber=1,
  numbersep=10pt,
  backgroundcolor=\color{lightblue},
  tabsize=2,
  linewidth=0pt,
  showspaces=false,
  showstringspaces=false,
  frame=single,
  framexleftmargin=10pt,
  framexrightmargin=10pt,
  framexbottommargin=7pt,
  framextopmargin=7pt,
  linewidth=350pt, % largeur de la ligne de code affichée
  xleftmargin=10pt, % espace avant le debut du cadre
  aboveskip=20pt
}

\usepackage[parfill]{parskip}             % Activate to begin paragraphs with an empty line rather than an indent

%%%%%%%%%%%%%%%%%%%%%%%%%%%%%%%%%%%%%%%%%%%%%%%%%%%%%%
%   Info sur le labo
%%%%%%%%%%%%%%%%%%%%%%%%%%%%%%%%%%%%%%%%%%%%%%%%%%%%%%
\newcommand{\branchetag}{WEB}
\newcommand{\branche}{Technologies web}
% \newcommand{\labonummer}{}
\newcommand{\laboname}{JQuery}
\newcommand{\auteurOne}{Romain de Wolff}
\newcommand{\auteurTwo}{Bruno da Silva}
\newcommand{\promo}{IL2008}

%%%%%%%%%%%%%%%%%%%%%%%%%%%%%%%%%%%%%%%%%%%%%%%%%%%%%%

\pagestyle{fancy} % defini nos propre header & footer
\fancyhf{} % delete current header and footer 
\fancyhead[L]{\branchetag}
\fancyhead[C]{\laboname}
\fancyhead[R]{\auteurOne \\ \auteurTwo} 
\fancyfoot[L]{\includegraphics[width=3cm]{img/logo-HEIG-VD.jpg}}
\fancyfoot[R]{\bfseries\thepage}

\renewcommand{\headrulewidth}{0.5pt} 
\renewcommand{\footrulewidth}{0.1pt} 
\addtolength{\footskip}{10.0pt} % space for the rule 
\fancypagestyle{plain}{
	\fancyhead{} % get rid of headers on plain pages 
	\fancyfoot{}
	\renewcommand{\headrulewidth}{0pt} % and the line 
	\renewcommand{\footrulewidth}{0pt} % and the line 
}

\author{\auteurOne, \auteurTwo}
\title{\branchetag : \laboname}
\date{\today}

\begin{document}
\pagenumbering{Roman}
\pagestyle{headings}
\begin{titlepage}
	\begin{center}
	\includegraphics{img/logo-HEIG-VD.jpg}\\
		\vspace{3cm}
		\LARGE \branche %Laboratoire No %\labonummer \\
		\vspace{3cm}\\
		\Huge \laboname \\
		\vspace{3cm}

		\Large \textsc{Tutorial} \\
		\vspace{3cm}

		\large \auteurOne \\
		\auteurTwo \\	
		\vspace{10pt}
		\normalsize \textsc{\promo} \\

		\vspace{2cm}
		\today
	\end{center}
\end{titlepage}

\tableofcontents
\newpage
\pagestyle{fancy}
\pagenumbering{arabic}
\section{Introduction}


\newpage

\section{Premiers pas}
% cf. http://techrageo.us/2007/07/05/jquery-introduction/

Nous allons commencer par créer un exemple très simple pour se plonger directement  au coeur de JQuery.

\subsection{Création d'une page HTML basique}

Pour commencer nous allons créer une page HTML basique de la forme suivante :

\begin{lstlisting}
<html>
<body>
  <p>Cliquer ce <a href="#" id="clickme">lien</a>.</p>
  <p class="lorem">Lorem ipsum dolor sit amet,
    consectetuer adipiscing elit.</p>
  <p>Ce texte va rester ici</p>
  <p class="lorem">Phasellus luctus. Suspendisse nisi
  	neque, feugiat eget, ullamcorper at, feugiat ut,
  	tellus.</p>
</body>
</html>
\end{lstlisting}

\subsection{Liaison du script JQuery}

Pour utiliser la librairie JQuery, nous allons devoir spécifier où se trouve ses sources. On peut, à choix, télécharger le fichier \emph{*.js} où alors mettre le chemin complet de son emplacement sur le site de JQuery. \\

Pour ce faire on rajoute une section \texttt{head} entre la section \texttt{html} et \texttt{body}.

\begin{lstlisting}
<head>
<script type="text/javascript"
  src="http://code.jquery.com/jquery-1.1.3.1.pack.js"/>
</head>
\end{lstlisting}

\subsection{Ajout de code standard JQuery}

Nous allons ajouter maintenant une fonction JQuery.

\begin{lstlisting}
	<script>
	$(document).ready(function() {
	  $("a#clickme").click( function() {
	    $("p.lorem").toggle(100);return false;
	  });
	});
	</script>
\end{lstlisting}

\subsection{Que ce passe-t-il?}

On a utilisé une classe HTML nommée \texttt{lorem} qui contient le texte qui disparaît. Le liens portant l'identifiant \texttt{clickme} permet de faire apparaître ou disparaître le texte. 

En définissant la fonction \texttt{toggle} sur click d'un lien, celui fait apparaître et  disparaître les éléments identifié grâce à la balise : 

\begin{lstlisting}
$("p.lorem").toggle(100);return false;")
\end{lstlisting}

On voit que \textbf{JQuery} permet d'une manière très simple d'accéder aux objet d'un document HTML, et ce, sans se soucier du navigateur utilisé.

\begin{lstlisting}
<script>
$(document).ready(function() {
  $("a#clickme").click( function() {
    $("p.lorem").toggle(100);return false;
  });
});
</script>
\end{lstlisting}

\subsection{Résumé}

Pour revenir sur les étapes effectuées, voilà en trois lignes ce que l'on a fait: \\

\begin{itemize}
	\item Définir/connaître le nom des objets dans la page HTML
	\item Lier le script JQuery à l'aide d'une des méthodes
	\item Ajouter une fonction sur un événement
\end{itemize}

Une liste exhaustives des événement est disponibles sur le site \url{http://www.gotapi.com}.

\section{}


\section{AJAX}

\newpage

\newpage
\section{Conclusion}

% \appendix
% 
% \newpage
% \section{CACMIndexer.java}
%\lstinputlisting{../src/rim/cacm/CACMIndexer.java}
% 
% \newpage
% \section{CACMRetriever.java}
% \lstinputlisting{../src/rim/cacm/CACMRetriever.java}

\end{document}