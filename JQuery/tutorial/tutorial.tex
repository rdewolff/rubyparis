%%%%%%%%%%%%%%%%%%%%%%%%%%%%%%%%%%%%%%%%%%%%%%%%%%%%%%%%%%%%%%%%%%%%%%%%%%%%%%%
% Auteurs : 			Romain de Wolff & Bruno da Silva
% Date de création : 	2 janvier 2008
%%%%%%%%%%%%%%%%%%%%%%%%%%%%%%%%%%%%%%%%%%%%%%%%%%%%%%%%%%%%%%%%%%%%%%%%%%%%%%%

\documentclass[10pt,a4paper,titlepage]{article}
\usepackage[utf8]{inputenc}     % encodage des characteres en utf8
\usepackage[francais]{babel} % pour la table des matières en français 

\usepackage{url} % pour les liens internet
\usepackage[colorlinks=true,linkcolor=black,bookmarks=true,bookmarksopen=true]{hyperref} % rendre les liens clickable
\usepackage{fancyhdr}	 

\usepackage[dvips]{graphicx}
\usepackage{epstopdf}


%%%%%%%%%%%%%%%%%%%%%%%%%%%%%%%%%%%%%%%%%%%%%%%%%%%%%%%%%%%%%%%%%%%%%%%%%%%%%%%
% Pour l'utilisation de code
%%%%%%%%%%%%%%%%%%%%%%%%%%%%%%%%%%%%%%%%%%%%%%%%%%%%%%%%%%%%%%%%%%%%%%%%%%%%%%%

\usepackage{listings} 
%\lstset{language=Java, breaklines, fontadjust, inputencoding=utf8, basicstyle=\small, numbers=left, numberstyle=\tiny, tabsize=2}

\usepackage{courier}
\usepackage{color}

% color definitions
\definecolor{dkgreen}{rgb}{0,0.6,0}
\definecolor{gray}{rgb}{0.5,0.5,0.5}
\definecolor{lightblue}{rgb}{0.92,0.92,1}

\lstset{language=Html,
  %keywords={break,case,catch,continue,else,elseif,end,for,function,
  %   global,if,otherwise,persistent,return,switch,try,while},
  basicstyle=\ttfamily\small,
  basicstyle=\scriptsize,
  keywordstyle=\color{blue},
  commentstyle=\color{dkgreen},
  stringstyle=\color{red},
  numbers=none,
  numberstyle=\tiny\color{gray},
  stepnumber=1,
  numbersep=10pt,
  backgroundcolor=\color{lightblue},
  tabsize=2,
  linewidth=0pt,
  showspaces=false,
  showstringspaces=false,
  frame=single,
  framexleftmargin=10pt,
  framexrightmargin=10pt,
  framexbottommargin=7pt,
  framextopmargin=7pt,
  linewidth=350pt, % largeur de la ligne de code affichée
  xleftmargin=10pt, % espace avant le debut du cadre
  aboveskip=20pt
}


\usepackage[parfill]{parskip}             % Activate to begin paragraphs with an empty line rather than an indent

%%%%%%%%%%%%%%%%%%%%%%%%%%%%%%%%%%%%%%%%%%%%%%%%%%%%%%%%%%%%%%%%%%%%%%%%%%%%%%%
%   Info sur le labo
%%%%%%%%%%%%%%%%%%%%%%%%%%%%%%%%%%%%%%%%%%%%%%%%%%%%%%%%%%%%%%%%%%%%%%%%%%%%%%%
\newcommand{\branchetag}{WEB}
\newcommand{\branche}{Technologies web}
% \newcommand{\labonummer}{}
\newcommand{\laboname}{JQuery}
\newcommand{\auteurOne}{Romain de Wolff}
\newcommand{\auteurTwo}{Bruno da Silva}
\newcommand{\promo}{IL2008}

%%%%%%%%%%%%%%%%%%%%%%%%%%%%%%%%%%%%%%%%%%%%%%%%%%%%%%%%%%%%%%%%%%%%%%%%%%%%%%%

\pagestyle{fancy} % defini nos propre header & footer
\fancyhf{} % delete current header and footer 
\fancyhead[L]{\branchetag}
\fancyhead[C]{\laboname}
\fancyhead[R]{\auteurOne \\ \auteurTwo} 
\fancyfoot[L]{\includegraphics[width=3cm]{img/logo-HEIG-VD.jpg}}
\fancyfoot[R]{\bfseries\thepage}

\renewcommand{\headrulewidth}{0.5pt} 
\renewcommand{\footrulewidth}{0.1pt} 
\addtolength{\footskip}{10.0pt} % space for the rule 
\fancypagestyle{plain}{
	\fancyhead{} % get rid of headers on plain pages 
	\fancyfoot{}
	\renewcommand{\headrulewidth}{0pt} % and the line 
	\renewcommand{\footrulewidth}{0pt} % and the line 
}

\author{\auteurOne, \auteurTwo}
\title{\branchetag : \laboname}
\date{\today}

\begin{document}
\pagenumbering{Roman}
\pagestyle{headings}
\begin{titlepage}
	\begin{center}
	\includegraphics{img/logo-HEIG-VD.jpg}\\
		\vspace{3cm}
		\LARGE \branche %Laboratoire No %\labonummer \\
		\vspace{3cm}\\
		\Huge \laboname \\
		\vspace{3cm}

		\Large \textsc{Tutorial} \\
		\vspace{3cm}

		\large \auteurOne \\
		\auteurTwo \\	
		\vspace{10pt}
		\normalsize \textsc{\promo} \\

		\vspace{2cm}
		\today
	\end{center}
\end{titlepage}

\tableofcontents

\newpage
\pagestyle{fancy}
\pagenumbering{arabic}
\section{Introduction}

Connaissance d'HTML requise.

\newpage

%%%%%%%%%%%%%%%%%%%%%%%%%%%%%%%%%%%%%%%%%%%%%%%%%%%%%%%%%%%%%%%%%%%%%%%%%%%%%%%
\section{Premiers pas}
%%%%%%%%%%%%%%%%%%%%%%%%%%%%%%%%%%%%%%%%%%%%%%%%%%%%%%%%%%%%%%%%%%%%%%%%%%%%%%%
% cf. http://techrageo.us/2007/07/05/jquery-introduction/

Nous allons commencer par créer un exemple très simple pour se plonger directement  au coeur de JQuery.

\subsection{Création d'une page HTML basique}

Pour commencer nous allons créer une page HTML basique de la forme suivante :

\begin{lstlisting}
<html>
<body>
  <p>Cliquer ce <a href="#" id="clickme">lien</a>.</p>
  <p class="lorem">Lorem ipsum dolor sit amet,
    consectetuer adipiscing elit.</p>
  <p>Ce texte va rester ici</p>
  <p class="lorem">Phasellus luctus. Suspendisse nisi
  	neque, feugiat eget, ullamcorper at, feugiat ut,
  	tellus.</p>
</body>
</html>
\end{lstlisting}

\subsection{Liaison du script JQuery}

Pour utiliser la librairie JQuery, nous allons devoir spécifier où se trouve ses sources. On peut, à choix, télécharger le fichier \emph{*.js} où alors mettre le chemin complet de son emplacement sur le site de JQuery. \\

Pour ce faire on rajoute une section \texttt{head} entre la section \texttt{html} et \texttt{body}.

\begin{lstlisting}
<head>
<script type="text/javascript"
  src="http://code.jquery.com/jquery-1.1.3.1.pack.js"/>
</head>
\end{lstlisting}

\subsection{Ajout de code standard JQuery}

Nous allons ajouter maintenant une fonction JQuery.

\begin{lstlisting}
	<script>
	$(document).ready(function() {
	  $("a#clickme").click( function() {
	    $("p.lorem").toggle(100);return false;
	  });
	});
	</script>
\end{lstlisting}

La ligne suivante écrite à l'aide de JQuery

\begin{lstlisting}
$(document).ready(function() { ... } )
\end{lstlisting}

aurait été écrite de la manière suivante en JavaScript  

\begin{lstlisting}
window.onload = function(){ ... }
\end{lstlisting}

\subsection{Que ce passe-t-il?}

On a utilisé une classe HTML nommée \texttt{lorem} qui contient le texte qui disparaît. Le liens portant l'identifiant \texttt{clickme} permet de faire apparaître ou disparaître le texte. 

En définissant la fonction \texttt{toggle} sur click d'un lien, celui fait apparaître et  disparaître les éléments identifié grâce à la balise : 

\begin{lstlisting}
$("p.lorem").toggle(100);return false;")
\end{lstlisting}

On voit que \textbf{JQuery} permet d'une manière très simple d'accéder aux objet d'un document HTML, et ce, sans se soucier du navigateur utilisé.

\begin{lstlisting}
<script>
$(document).ready(function() {
  $("a#clickme").click( function() {
    $("p.lorem").toggle(100);return false;
  });
});
</script>
\end{lstlisting}

\subsection{Résumé}

Pour revenir sur les étapes effectuées, voilà en trois lignes ce que l'on a fait: \\

\begin{itemize}
	\item Définir/connaître le nom des objets dans la page HTML
	\item Lier le script JQuery à l'aide d'une des méthodes
	\item Ajouter une fonction sur un événement
\end{itemize}

Une liste exhaustives des événement est disponibles sur le site \url{http://www.gotapi.com}.

\subsection{Sélectionner d'autres éléments}

Voici quelques autres exemples de sélection à l'aide de JQuery. 

\begin{lstlisting}
jQuery('div.panel')
\end{lstlisting}

Tous les éléments \texttt{divs} avec \texttt{class="panel"}

\begin{lstlisting}
jQuery('p#intro')
\end{lstlisting}

Le paragraphe avec \texttt{id="intro"}

\begin{lstlisting}
jQuery('div#content a:visible')
\end{lstlisting}

Tous les liens visibles qui se trouvent à l'intérieur du \texttt{div} avec \texttt{id="content"}

\begin{lstlisting}
jQuery('input[@name=email]')
\end{lstlisting}

Tous les champs de formulaire comportant \texttt{name="email"}

\begin{lstlisting}
jQuery('table.orders tr:odd')
\end{lstlisting}

Les lignes numérotée de tables “étrange” comportant la classe \texttt{orders}

\begin{lstlisting}
jQuery('a[@href^="http://"]')
\end{lstlisting}

Tous les liens externes (liens qui commencent par \verb!http://!).

\begin{lstlisting}
jQuery('p[a]')
\end{lstlisting}

Tous les paragraphes qui contienne un ou plusieurs liens.


\newpage
%%%%%%%%%%%%%%%%%%%%%%%%%%%%%%%%%%%%%%%%%%%%%%%%%%%%%%%%%%%%%%%%%%%%%%%%%%%%%%%
\section{AJAX}
%%%%%%%%%%%%%%%%%%%%%%%%%%%%%%%%%%%%%%%%%%%%%%%%%%%%%%%%%%%%%%%%%%%%%%%%%%%%%%%

AJAX permet d'exécuter des requêtes sur d'autre pages sans recharger tout le document en cours. Il permett donc de transférer des données ou de pré-charger des données. JQuery permet de faire du AJAX simple de manière extrêmement facile, et permet de rendre les choses complexe tout à fait faisables. 

Voici quelques exemples de l'utilisation d'AJAX à l'aide de JQuery.

\subsection{Charger des données dans une zone}

Une utilisation classique de Ajax est de charger un bout de code HTML à l'intérieur d'une zone d'une page.

Pour faire ceci à l'aide de JQuery, il suffit de sélectionner l'élément désiré et d'utiliser la fonction \texttt{load()}, comme montré dans l'exemple ci dessous 

\begin{lstlisting}
$('#stats').load('stats.html');
\end{lstlisting}


\subsection{Passer des paramètres a une page}

Très souvent on veut simplement passer des paramètres a une page sur le serveur. Comme vous vous y attendez, ceci se fait de manière extrêmement simple à l'aide de JQuery. 

On a le choix entre \texttt{\$.post()} et \texttt{\$.get()} en fonction de la méthode dont on a besoin. On peut ensuite spécifier optionnellement un objet comportant les données ainsi qu'une fonction appelée une fois l'opération terminée (fonction de type \emph{callback}).

\begin{lstlisting}
$.post('save.cgi', {
    text: 'my string',
    number: 23
}, function() {
    alert('Your data has been saved.');
});
\end{lstlisting}
                                   
\subsection{Instruction complexes en AJAX rendu simple}

Si l'on désire vraiment effectuer des opérations complexes en AJAX, il faut utiliser la fonction \texttt{\$.ajax()}. On spécifie le type de données utilisée (\texttt{xml, html, script ou json}) et JQuery va préparer les résultats de la fonction de retour automatiquement afin de pouvoir l'utiliser tout de suite. 

On peut aussi spécifier \texttt{beforeSend}, \texttt{error}, \texttt{success} ou encore \texttt{complete} qui permet de mieux gérer ce qu'il se passe durant l'utilisation d'AJAX. Pour par exemple dire à l'utilisateur que la transaction a commencée et l'avertir quand elle est terminée.

D'autre paramètre sont disponible, comme \texttt{timeout} qui permet de limiter le temps d'exécution d'une requête en AJAX.

Voici un exemple d'utilisation d'AJAX. 

\begin{lstlisting}
$.ajax({
    url: 'document.xml',
    type: 'GET',
    dataType: 'xml',
    timeout: 1000,
    error: function(){
        alert('Error loading XML document');
    },
    success: function(xml){
        // faire quelques chose des donnees xml
    }
});
\end{lstlisting}

On peut ensuite imaginer d'utiliser encore JQuery pour traiter les données XML reçues et les afficher. Pour ce faire, on va utiliser le même principe que pour les fichiers HTML. Cela rends facile de parcourir un fichier XML et d'en afficher le contenu dans un fichier HTML.

L'exemple suivant montre comment utiliser le paramètre \texttt{sucess} pour appeler une fonction d'affichage de notre XML une fois la transaction correctement effectuées.

\begin{lstlisting}
success: function(xml){
    $(xml).find('item').each(function(){
        var item_text = $(this).text();

        $('<li></li>')
            .html(item_text)
            .appendTo('ol');
    });
}
\end{lstlisting}

\section{Animations}




\newpage
%%%%%%%%%%%%%%%%%%%%%%%%%%%%%%%%%%%%%%%%%%%%%%%%%%%%%%%%%%%%%%%%%%%%%%%%%%%%%%%
\section{Conclusion}
%%%%%%%%%%%%%%%%%%%%%%%%%%%%%%%%%%%%%%%%%%%%%%%%%%%%%%%%%%%%%%%%%%%%%%%%%%%%%%%

Nous espérons que nous avons pus vous montrer que JQuery n'est pas une simple nouvelle librairie JavaScript comme il en existe tant. 

% \appendix
% 
% \newpage
% \section{CACMIndexer.java}
%\lstinputlisting{../src/rim/cacm/CACMIndexer.java}
% 
% \newpage
% \section{CACMRetriever.java}
% \lstinputlisting{../src/rim/cacm/CACMRetriever.java}

\end{document}

%%%%%%%%%%%%%%%%%%%%%%%%%%%%%%%%%%%%%%%%%%%%%%%%%%%%%%%%%%%%%%%%%%%%%%%%%%%%%%%

\lstdefinelanguage{JavaScript}{
     keywords={attributes, class, classend, do, empty, endif, endwhile, fail, function, functionend, if, implements, in, inherit, inout, not, of, operations, out, return, set, then, types, while, use},
     keywordstyle=\color{blue}\bfseries,
     ndkeywords={},
     ndkeywordstyle=\color{yellow}\bfseries,
     identifierstyle=\color{black},
     sensitive=false,
     comment=[l]{//},
     commentstyle=\color{green}\ttfamily,
     stringstyle=\color{red}\ttfamily
  }

  \lstset{
     language=JavaScript,
     extendedchars=true,
     basicstyle=\scriptsize,
     showstringspaces=false,
     numbers=left,
     numberstyle=\tiny,
     stepnumber=1,
     numbersep=5pt,
  }
