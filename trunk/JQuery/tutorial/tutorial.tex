\documentclass[10pt,a4paper,titlepage]{article}
\usepackage[utf8]{inputenc}     % encodage des characteres en utf8
\usepackage[francais]{babel} % pour la table des matières en français 

\usepackage{url} % pour les liens internet
\usepackage[colorlinks=true,linkcolor=black,bookmarks=true,bookmarksopen=true]{hyperref} % rendre les liens clickable
\usepackage{fancyhdr}	 
\usepackage{listings} 
\lstset{language=Java, breaklines, fontadjust, inputencoding=utf8, basicstyle=\small, numbers=left, numberstyle=\tiny, tabsize=2}

\usepackage[dvips]{graphicx}
\usepackage{epstopdf}


%%%%%%%%%%%%%%%%%%%%%%%%%%%
%   Info sur le labo
%%%%%%%%%%%%%%%%%%%%%%%%%%%
\newcommand{\branchetag}{WEB}
\newcommand{\branche}{Technologies web}
% \newcommand{\labonummer}{}
\newcommand{\laboname}{JQuery}
\newcommand{\auteurOne}{Romain de Wolff}
\newcommand{\auteurTwo}{Bruno da Silva}
\newcommand{\promo}{IL2008}

%%%%%%%%%%%%%%%%%%%%%%%%%%%

\pagestyle{fancy} % defini nos propre header & footer
\fancyhf{} % delete current header and footer 
\fancyhead[L]{\branchetag}
\fancyhead[C]{\laboname}
\fancyhead[R]{\auteurOne \\ \auteurTwo} 
\fancyfoot[L]{\includegraphics[width=3cm]{img/logo-HEIG-VD.jpg}}
\fancyfoot[R]{\bfseries\thepage}

\renewcommand{\headrulewidth}{0.5pt} 
\renewcommand{\footrulewidth}{0.1pt} 
\addtolength{\footskip}{10.0pt} % space for the rule 
\fancypagestyle{plain}{
	\fancyhead{} % get rid of headers on plain pages 
	\fancyfoot{}
	\renewcommand{\headrulewidth}{0pt} % and the line 
	\renewcommand{\footrulewidth}{0pt} % and the line 
}

\author{\auteurOne, \auteurTwo}
\title{\branchetag : \laboname}
\date{\today}

\begin{document}
\pagenumbering{Roman}
\pagestyle{headings}
\begin{titlepage}
	\begin{center}
	\includegraphics{img/logo-HEIG-VD.jpg}\\
		\vspace{3cm}
		\LARGE \branche %Laboratoire No %\labonummer \\
		\vspace{3cm}\\
		\Huge \laboname \\
		\vspace{3cm}

		\Large \textsc{Tutorial} \\
		\vspace{3cm}

		\large \auteurOne \\
		\auteurTwo \\	
		\vspace{10pt}
		\normalsize \textsc{\promo} \\

		\vspace{2cm}
		\today
	\end{center}
\end{titlepage}

\tableofcontents
\newpage
\pagestyle{fancy}
\pagenumbering{arabic}
\section{Introduction}


\newpage
\section{Premiers pas}
% cf. http://techrageo.us/2007/07/05/jquery-introduction/

Nous allons commencer par créer un exemple très simple pour se plonger directement  au coeur de JQuery.

\subsection{Création d'une page HTML basique}

Pour commencer nous allons créer une page HTML basique de la forme suivante :

\begin{verbatim}
<html>
<body>
  <p>Cliquer ce <a href="#" id="clickme">lien</a>.</p>
  <p class="lorem">Lorem ipsum dolor sit amet,
    consectetuer adipiscing elit.</p>
  <p>Ce texte va rester ici</p>
  <p class="lorem">Phasellus luctus. Suspendisse nisi
  	neque, feugiat eget, ullamcorper at, feugiat ut,
  	tellus.</p>
</body>
</html>
\end{verbatim}

\subsection{Liaison du script JQuery}

Pour utiliser la librairie JQuery, nous allons devoir spécifier où se trouve ses sources. On peut, à choix, télécharger le fichier \emph{*.js} où alors mettre le chemin complet de son emplacement sur le site de JQuery. \\

Pour ce faire on rajoute une section \texttt{head} entre la section \texttt{html} et \texttt{body}.

\begin{verbatim}
	<head>
	  <script type="text/javascript"
	    src="http://code.jquery.com/jquery-1.1.3.1.pack.js"/>
	</head>
\end{verbatim}

\subsection{Ajout de code standard JQuery}

Nous allons ajouter maintenant une fonction JQuery.

\begin{verbatim}
	<script>
	$(document).ready(function() {
	  $("a#clickme").click( function() {
	    $("p.lorem").toggle(100);return false;
	  });
	});
	</script>
\end{verbatim}

\subsection{Que ce passe-t-il?}

On a utilisé une classe HTML nommée \texttt{lorem} qui contient le texte qui disparaît. Le liens portant l'identifiant \texttt{clickme} permet de faire apparaître ou disparaître le texte. 

En définissant la fonction \texttt{toggle} sur click d'un lien, celui fait apparaître et  disparaître les éléments identifié grâce à la balise : \\

$\$("p.lorem").toggle(100);return false;")$ \\ 

On voit que JQuery permet d'une manière très simple d'accéder aux objet d'un document HTML, et ce, sans se soucier du navigateur utilisé.


\newpage

\newpage
\section{Conclusion}

% \appendix
% 
% \newpage
% \section{CACMIndexer.java}
% \lstinputlisting{../src/rim/cacm/CACMIndexer.java}
% 
% \newpage
% \section{CACMRetriever.java}
% \lstinputlisting{../src/rim/cacm/CACMRetriever.java}

\end{document}